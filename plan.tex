\documentclass{scrreprt}
% coma version of report class:
% http://tex.stackexchange.com/questions/5948/subtitle-doesnt-work-in-article-document-class

\usepackage{fullpage}
\usepackage[utf8]{inputenc} % åäö
\usepackage{hyperref}
\usepackage{enumitem}

\setkomafont{disposition}{\normalfont\bfseries}


% Remove date
\date{2014}

\hypersetup{
  colorlinks = true,
  linkcolor = black,
  citecolor = red
}

\title{ A language for defining markup languages }
\subtitle{ Using regular expressions to convert annotated documents \\into hierarchical documents }
\author{ Christopher OKHRAVI \\ UPPSALA UNIVERSITY }


%
% Defining research questions
%
\newcommand\researchquestionformat[1]{\begin{quote}#1\end{quote}}
\newcommand\firstresearchquestion{\researchquestionformat{%
  \textbf{(RQ1) Research question 1} \\
  Can the verbosity of document annotation formats (e.g. \texttt{XML}) be decreased, by allowing authors themselves to design annotation formats?%
}}


\newcommand\secondresearchquestion{\researchquestionformat{%
  \textbf{(RQ2) Research question 2} \\
  Can a \texttt{Domain Specific Language (DSL)} be defined, such that a semi-technical author can employ it to design a document annotation format?

  Requiring (A) the combined complexity of the DSL and the annotated document is less than that of a document annotated in a traditional markup language with fixed syntax (e.g. \texttt{XML}).

  Without (B) sacrificing any flexibility of the original markup language.
}}

\newcommand\thirdresearchquestion{\researchquestionformat{%
  \textbf{(RQ3) Research question 3} \\
  Can a process be defined, such that any document can be converted into \texttt{XML} format, using \texttt{Regular Expressions} as rules for document hierarchy?
}}

\newcommand\fourthresearchquestion{\researchquestionformat{%
  \textbf{(RQ4) Research question 4} \\
  Can a language be defined, such that it is a subset of \texttt{Regular Expressions}, where most control characters are replaced by assumptions? Where the intent of the subset is to express annotations of document hierarchy.
}}





\begin{document}

\maketitle
\tableofcontents
\pagebreak



\chapter{Introduction}
Lorem ipsum.

\section{Research questions}
The question of interest is as follows.

\firstresearchquestion

Given that there are a multitude of potentials solutions to the question, one path must be chosen. This thesis concerns itself with the approach outlined in the more detailed question below.

\secondresearchquestion

To explore the possibility of the above question the author has chosen to evaluate whether \texttt{Regular Expressions} can be used to define annotation rules that enables an annotated document to be converted into \texttt{XML}. As expressed by the question below:

\thirdresearchquestion

Given that \texttt{Regular Expressions} may cause the level of complexity to increase (which previously stated as explicitly unwanted) the author has chosen to suggest the use of a simplified subset of the language. As expressed in the question below:

\fourthresearchquestion


\section{Delimitations}
Lorem ipsum.




\chapter{Method}
Lorem ipsum.

\section{Data analysis}
Lorem ipsum.





\chapter{Literature review}
Lorem ipsum.

\section{Annotating documents}
What does it mean to annotate? What are common document annotation formats?

\section{Object notation as a subset of annotation}
An explanation as to why object notation is a subset of annotations. Meaning that formats like \texttt{JSON} are too data-centric and can consequently not, in any sensible manner, be used for manual document authoring.

\section{Common markup languages}
Exploring a comprehensive list of existing markup languages, including light-weight flavors such as markdown.

\section{Semantic coupling}
Exploring the level of semantic coupling between the annotation format and it's intended output.





\chapter{Empirics}
Lorem ipsum.


\section{SimEx -- A control structure-light subset of regex}
...


\section{Flexup -- A DSL for expressing regex to XML conversions}
...

\subsection{The annotated file (\texttt{.fup})}
...

\subsection{The annotation definitions file (\texttt{.fupd})}
...

\subsection{The binary}
...





\chapter{Analysis}
Lorem ipsum.





\chapter{Conclusion}
Lorem ipsum.





\chapter{Discussion}
Lorem ipsum.

\section{Future research}
Lorem ipsum.






\end{document}