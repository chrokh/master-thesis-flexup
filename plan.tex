\documentclass{scrreprt}
% coma version of report class:
% http://tex.stackexchange.com/questions/5948/subtitle-doesnt-work-in-article-document-class

\usepackage{fullpage}
\usepackage[utf8]{inputenc} % åäö
\usepackage{hyperref}
\usepackage{enumitem}
\usepackage[toc]{glossaries}



\setkomafont{disposition}{\normalfont\bfseries}


% Remove date
\date{2014}

\hypersetup{
  colorlinks = true,
  linkcolor = black,
  citecolor = red
}

\title{ A language for defining markup languages }
\subtitle{ Using regular expressions to convert annotated documents \\into hierarchical documents }
\author{ Christopher OKHRAVI \\ UPPSALA UNIVERSITY }


%
% Defining research questions
%
\newcommand\researchquestionformat[1]{\begin{quote}#1\end{quote}}
\newcommand\firstresearchquestion{\researchquestionformat{%
  \textbf{(RQ1) Research question 1} \\
  Can the verbosity of document annotation formats (e.g. \texttt{XML}) be decreased, by allowing authors themselves to design annotation formats?%
}}


\newcommand\secondresearchquestion{\researchquestionformat{%
  \textbf{(RQ2) Research question 2} \\
  Can an annotation language
  For an author to be able to design an annotation language, this author must design a domain-specific-language (DSL), a parser, a compiler and a transformation engine. Consequently we come to the point where we instead ask ourselves the following question:


  Can a \texttt{Domain Specific Language (DSL)} be defined, such that a semi-technical author can employ it to design a document annotation format?

  Requiring (A) the combined complexity of the DSL and the annotated document is less than that of a document annotated in a traditional markup language with fixed syntax (e.g. \texttt{XML}).

  Without (B) sacrificing any flexibility of the original markup language.
}}

\newcommand\thirdresearchquestion{\researchquestionformat{%
  \textbf{(RQ3) Research question 3} \\
  Can a process be defined, such that any document can be converted into \texttt{XML} format, using \texttt{Regular Expressions} as rules for document hierarchy?
}}

\newcommand\fourthresearchquestion{\researchquestionformat{%
  \textbf{(RQ4) Research question 4} \\
  Can a language be defined, such that it is a subset of \texttt{Regular Expressions}, where most control characters are replaced by assumptions? Where the intent of the subset is to express annotations of document hierarchy.
}}




\newglossaryentry{document authoring}
{
  name={Document Authoring},
  description={The act of writing literature.}
}
\makeglossaries






\begin{document}

\maketitle
\tableofcontents
\pagebreak



\glsaddall
\printglossary





\chapter{Introduction}
- A brief history of writing from pens, to the printing press, to computers and word processors, to markup languages.
 
- A brief overview of markup languages and how they can be used for (1) Document authoring, and (2) Data transport.

- Exploring the problem of verbosity in markup languages, and leading the reader towards the area of research.



\section{Research questions}
Let us formulate this problem as a research question.

\firstresearchquestion

Let's break down what it means to be able to work with an arbitrarily annotated document. Initially, the author would of course have to decide upon an annotation syntax. If the syntax doesn't look like any existing annotation syntax we immediately run into a problem. To be able to convert our arbitrary syntax into any other structured format (such as a \texttt{PDF} or a \texttt{HTML} document) we would need a parser that can ``read and treat'' the document as a tree\footnote{ We'll return to discuss why free text documents (regardless of annotation syntax) can be treated as a trees.}.

Assume we've created a parser that can ``read and treat'' an arbitrary document written in our syntax. Assume we've written this parser in any programming language of your choice. Assume the parser, written in your programming language of choice, builds up the document tree in memory. Now obviously the next problem arise. A document tree in memory does not do us any good if we cannot export this tree from memory into any other format of our wishing. This means we've only covered the first subproblem of two. We'll express this second subproblem as -- converting the tree representation of the document from memory into any given concrete format of our choosing, that in turn can be written to disk.

In summary -- the first subproblem is an ``interpretation problem'', whereas the second a ``transformation problem''. Without making any assumptions about whether the tools exist or not -- let us formalize these two as requirements for tools: For an author to be able to work with an arbitrary annotation syntax one would need:

\begin{enumerate}
\item A parser (that ``reads the annotated document as a tree''),
\item A transformation engine (that ``converts the read document into a tree'').
\end{enumerate}

In fact, the problem stated above is not actually that complicated. In fact, given a moderately skilled programmer, the problem can easily be solved today -- without introducing any new tools. Let's look at an example.

Assume a document author is approaching this problem. Assume that the author in question, after designing a custom annotation syntax, constructs a set of \texttt{Regular Expressions} that utilizes captures to distinguish data from annotations. Assume then that the author would apply these regular expressions to the document (written in the custom annotation syntax)  using one of the few engines that convert \texttt{regex captures} to \texttt{XML}. Perhaps the author might even write a custom \texttt{regex} to \texttt{XML} converter -- given the lack of existence\footnote{ Search strategies are declared in the method section of the thesis.} of a of a ``de facto'' standardized conversion strategy\footnote{ The conversion is not trivial, and thus many different solutions may be considered ``correct''. Further investigation is found in the empirical sections of this thesis.}.

Given that \texttt{regex} is a widely used (\#REF!), and perhaps the most common, way of searching plain-text (\#REF!), it is a fitting technique for identifying annotations in semi-structured text. Further, \texttt{XML} is a widely used (\#REF!), and perhaps the most common, way of representing documents as hierarchies of information, where annotations are easily separated from data.

Further, \texttt{XML} is a suitable output format because of it being free from semantics. Put this in contrast to for example \texttt{HTML} or \texttt{LaTeX} which both have semantic properties embedded into their specifications. Consider for example the paragraph element in \texttt{HTML} (\texttt{<p>..</p>}). The element denotes semantic significance to the contents inside it. It denotes a paragraph. The same problem arise when using, previously mentioned, \texttt{LaTeX}. According to the the project documentation it is a ``typesetting system [...] for producing [...] documents'' \footnote{ http://www.techscribe.co.uk/ta/latex-introduction.pdf}. Semantically, it is tightly bound to the ``idea'' of the physical document (\#REF!). 

Given that \texttt{XML} lives in an eco-system with other useful languages and tools -- such as \texttt{XSL}, \texttt{XSLT}, \texttt{XPath}, \texttt{DOM} etc. -- it has also proven to be a useful data transportation format (\#REF!). Meaning that if two given systems need to share data, \texttt{XML} can serve as a neutral intermediate format.

Consequently, the idea is to use \texttt{regex}:es to parse an arbitrarily (but known) annotated document, and convert the parsed result into \texttt{XML}. For the purpose of allowing an author to turn an arbitrary annotation syntax into virtually any other structured format. Using \texttt{XML} as an intermediate format, and \texttt{regex} as a means of parsing.

While this all seems reasonable, we have yet not to address the question of why this approach would need any further research. Given that the two above discussed techniques are both well documented and well used -- it might seem this technique is highly employable today.

\begin{enumerate}
\item design/use a parser (that ``reads the annotated document''),
\item design/use a compiler (that ``converts the read document into a tree''),
\item design/use a transformation engine.
\end{enumerate}


Streamlining,   argue  Consequently we come to the point where we instead ask ourselves the following question:

\secondresearchquestion

To explore the possibility of the above question the author has chosen to evaluate whether \texttt{Regular Expressions} can be used to define annotation rules that enables an annotated document to be converted into \texttt{XML}. As expressed by the question below:

\thirdresearchquestion

Given that \texttt{Regular Expressions} may cause the level of complexity to increase (which previously stated as explicitly unwanted) the author has chosen to suggest the use of a simplified subset of the language. As expressed in the question below:

\fourthresearchquestion


\section{Delimitations}
Given that markup languages are prolifically used in areas other than document authoring (such as data transportation), it is naïve to not recognize that an effort in one may allow (or force) change in the other(s). While the author do admit this multifaceted nature of markup languages, \emph{all facets of markup languages expect document authoring are considered outside the scope of this thesis}. In the literature review these other use cases will be touched upon, only as to provide the reader with context. Any effects that may ripple out of the results and into other usage areas of markup languages will within this thesis be ignored. 




\chapter{Method}



\section{Data analysis}
Lorem ipsum.





\chapter{Literature review}
Lorem ipsum.

\section{Annotating documents}
What does it mean to annotate? What are common document annotation formats?

\section{Object notation as a subset of annotation}
An explanation as to why object notation is a subset of annotations. Meaning that formats like \texttt{JSON} are too data-centric and can consequently not, in any sensible manner, be used for manual document authoring.

\section{Common markup languages}
Exploring a comprehensive list of existing markup languages, including light-weight flavors such as markdown.

\section{Semantic coupling}
Exploring the level of semantic coupling between the annotation format and it's intended output.





\chapter{Empirics}
Lorem ipsum.


\section{SimEx -- A control structure-light subset of regex}
...


\section{Flexup -- A DSL for expressing regex to XML conversions}
...

\subsection{The annotated file (\texttt{.fup})}
...

\subsection{The annotation definitions file (\texttt{.fupd})}
...

\subsection{The binary}
...





\chapter{Analysis}
Lorem ipsum.





\chapter{Conclusion}
Lorem ipsum.





\chapter{Discussion}
Lorem ipsum.

\section{Future research}
Lorem ipsum.






\end{document}